\documentclass[../MasterThesis.tex]{subfiles}
\graphicspath{ {./assets/images/} }


%----------------------------------------------------------------------------
%----------------------------------------------------------------------------

\begin{document}
	
	
	

%
%
%
%
%=======================================================================================================
%
%
%
%
%=======================================================================================================
% CHAPTER: SYSTEM REQUIREMENTS AND SPECIFICATIONS
%=======================================================================================================
\newpage
\section{System Requirements and Specifications} \label{section:systemrequirementsandspecifications}

% TODO Introduction to System Requirements

%Define the overall goals and objectives of your project.
%Specify the functional and non-functional requirements of the system. This includes what the system is expected to do, its features, and any constraints it must adhere to.
%Outline any specific hardware or software requirements necessary for the project.
%Detail the methodologies or approaches you'll use for system development.



%-------------------------------------------------------------------------------------------------------
%\subsection{User Requirements} \label{subsection:userrequirements}
% Identify and describe the requirements from the end-user perspective.



%Who are the primary users of the system?
%Example: "Are the primary users of the system employees in a particular department, such as sales representatives, managers, or administrators?"


%User Goals and Tasks:
%What are the main goals and objectives of the users when interacting with the system?
%Example: "Do sales representatives need to use the system to manage customer information, create sales orders, and generate reports?"


%User Environment:
%Where and how will users typically access the system?
%Example: "Will users primarily access the system from desktop computers in the office, or do they need mobile access while on the go?"


%User Experience Preferences:
%What are the usability expectations or preferences of the users regarding interface design, navigation, etc.?
%Example: "Are users accustomed to working with similar software applications, and do they have any specific preferences for interface layouts or menu structures?"

%User Skill Levels:
%What are the technical skills and knowledge levels of the users?
%Example: "Are users expected to have prior experience with similar software applications, or will they require training or documentation to use the system effectively?"


%User Feedback and Suggestions:
%Have users provided any feedback or suggestions for system improvements?
%Example: "Have users identified any pain points or inefficiencies in their current workflows that the new system should address?"


%User Accessibility Needs:
%Are there any accessibility requirements that need to be addressed for users with disabilities?
%Example: "Do any users have specific accessibility needs, such as support for screen readers or keyboard navigation for users with visual impairments?"


%Cultural or Language Considerations:
%Are there any cultural or language considerations that need to be accommodated for the users?
%Example: "Will the system need to support multiple languages or cultural conventions to accommodate users in different regions or countries?"






%-------------------------------------------------------------------------------------------------------
%\subsection{Functional Requirements} \label{subsection:functionalrequirements}








%-------------------------------------------------------------------------------------------------------
%\subsection{Non-Functional Requirements} \label{subsection:nonfunctionalrequirements}






In this Chapter, the technical requirements for the system are described. These requirements provide the necessary standards to ensure the systems functionality. Additionally, the execution of the code is described.




%-------------------------------------------------------------------------------------------------------
\subsection{Technical Requirements} \label{subsection:technicalrequirements}
% Detail the technical requirements for the system.



Regarding the hardware requirements, the system should be capable of running Node.js, which is necessary for executing the frontend development tool Yarn. Additionally, the system should be able to execute the python backend and Melt framework within a Docker container.
While implementing, it was observed that the system should have a minimum of $16$ GB RAM to ensure smooth execution of the development environment and Docker containers.



The software requirements include information about the operating system, applications and frameworks.
While testing, the code encountered difficulties running on Windows. While it should be possible that with debugging and configuration adjustments it could run, it was decided to proceed with this thesis project on a Linux system (Ubuntu 22.04). This decision was influenced by factors such as the compatibility of the code base with Linux and its command-line interface.
Regarding the documentation in the README files, the Mac operating system should be able to run the code after making adjustments, but this was not tested in the scope of this thesis.


In addition to this, the system relies on several software components.
Docker is needed to use Docker containers for encapsulating the system's backend. This allows consistent and efficient results across different devices.
Node.js serves as the runtime environment for executing JavaScript code for web applications and Yarn is the package manager for Node.js applications, that is used in this project.
Additionally, Git is essential for version control. While it is not strictly required for the system's execution, Git is used to pull Git repositories containing the code and necessary dependencies and frameworks, such as the Melt framework. 
An installation of the Melt framework is not needed for the execution or development of this system, but can be useful for testing or exploring the functionalities of Melt.
%
%Regarding the security measurements, the Docker containers use a \textit{you snooze you lose} approach, where the container terminates automatically if there are no incoming requests within a certain time frame. 
Furthermore, the system adheres to standard security practices.


	
	
	
	
	
	








% \newpage
%-------------------------------------------------------------------------------------------------------
\subsection{Execution of the Code} \label{subsection:runninghtecode}
% Discuss the technologies and tools chosen for the implementation.

The codebase for this project consists of two directories -- \texttt{accurate-player-3-core} and \texttt{jit-webrtc}. The system's architecture is described in Section~\ref{section:designandimplementation}. Additionally, the READMEs of the two main components of the system can be seen in Appendix~\ref{appendix:readme}.


The files for the frontend code are in a directory that is called \texttt{accurate-player-3-core}. To execute the frontend, it needs to be navigated to the directory \texttt{demo} in \texttt{packages} and then \texttt{yarn start} needs to be executed with the parameter for the backend in the CLI:
\begin{lstlisting}[language=bash, numbers=none, columns=fullflexible]
	~/accurate-player-3-core/packages/demo$ JIT_BACKEND=http://localhost:8080 yarn start
\end{lstlisting}

For updating dependecies during development, the following command should be executed and running in addition: 
\begin{lstlisting}[language=bash, numbers=none, columns=fullflexible]
	~/accurate-player-3-core/packages/jit$ yarn start
\end{lstlisting}



To run the backend, the folder \texttt{jit-webrtc} needs to be accessed in the CLI to start the script \texttt{main.sh}, that starts the backend in a docker container and pulls the needed dependencies:
\begin{lstlisting}[language=bash, numbers=none, columns=fullflexible]
	~/jit-webrtc$ docker/main/main.sh -v --threads 16 --port 8080  https://s3.eu-central-1.amazonaws.com/accurate-player-demo-assets/timecode/sintel-2048-timecode-stereo.mp4
\end{lstlisting}

The optional parameter \texttt{-v} (verbose) provides additional information or detailed output when executing the code. The threads and the port are set with the parameters \texttt{--threads} and \texttt{--port} and then the link to the test video file is added. This is the file that will then be played in the Accurate Player.






	
	
	

	
	
\end{document}