\documentclass[../MasterThesis.tex]{subfiles}
\graphicspath{ {./assets/images/} }


%----------------------------------------------------------------------------
%----------------------------------------------------------------------------

\begin{document}

	%
	%
	%=======================================================================================================
	%
	%
	%
	%
	%=======================================================================================================
	% CHAPTER: INTRODUCTION
	%=======================================================================================================
	\newpage
	\section{Introduction} \label{section:introduction}
	
	
	The popularity of streaming is growing, with an increasing number of people using online streaming platforms for their entertainment and approximately 1.8 billion subscriptions to video streaming services.~\cite{nielsen, stats}
	%
	With this increased popularity, the demand for user-driven real-time visual customization in the web browser for streamed videos has also increased.
	
	
	
	
	
	
	
	
	
	
	
	
	
	
	
	
	
	
	%-------------------------------------------------------------------------------------------------------
	\subsection{Motivation} \label{subsection:motivation}
	
	
	The colouring of a video can evoke different emotions and set an overall tone to the perception of the content. For instance, cool colours like blue can convey cold weather, while warm colours like yellow can create a feeling of warmth and sunlight. 
	The colouring of a video is a powerful tool that can significantly impact the conveyed mood and atmosphere.~\cite{colorgrading1, colorgrading2}
	
	
	
	
	\begin{figure}[H]
		\begin{center}
			\cutpic{0.3cm}{0.3\textwidth}{2_blau.jpg}
			%\hspace*{0.01\textwidth}
			\cutpic{0.3cm}{0.3\textwidth}{2.jpg}
			%\hspace*{0.01\textwidth}
			\cutpic{0.3cm}{0.3\textwidth}{2_gelb.jpg}
			\caption[Photo in different colour tones]{The original photo can be seen in the middle, while the picture on the left has more blue tones and the picture on the right more yellow tones.}
			\label{figure:coloursblueandyellow}
		\end{center}
	\end{figure}
	%
	%
	%
	The use-cases for on-the-fly adjustments of the video colour arise from different reasons. 
	One of them is that each screen, ranging from computer monitors, TVs, smartphones and tablets to beamers, has unique characteristics that can influence the perceived colour palette. 
	In addition to this, the light conditions can influence the appearance of the colours.
	With real-time colour corrections, the user can adjust the colour and cancel out variations that are caused by the device or environment to have a consistent experience.
	
	Real-time colour grading is a solution that not only addresses the variability in device displays but also enables users to adjust the image according to their preferences.
	
	Additionally, users with visual problems or specific colour perception might benefit from real-time colour grading to enhance the visibility and distinction of the on-screen elements. 
	The increased accessibility can contribute to enhanced inclusivity and enable that a diverse range of users can engage with the content.
	
	
	
	
	%1. **Device Variability:** As mentioned earlier, different devices have distinct display characteristics. Users may prefer to adjust colours to compensate for variations in brightness, contrast, or colour rendering among devices.
	
	%2. **Personal Preferences:** Everyone perceives colours differently, and personal preferences for visual aesthetics vary. Real-time colour correction empowers users to customize the viewing experience according to their individual tastes.
	
	%3. **Environmental Conditions:** Lighting conditions in the viewing environment can impact how colours appear on the screen. Real-time colour correction allows users to adapt to different lighting situations, ensuring optimal visibility and comfort.
	
	%4. **Accessibility Considerations:** Users with visual impairments or specific color perception needs may benefit from real-time colour correction to enhance visibility and distinguishability of on-screen elements.
	
	
	%6. **Cultural Differences:** Cultural preferences and associations with colours can vary globally. Real-time colour correction accommodates diverse cultural perspectives, allowing users to align the visual presentation with their cultural preferences.
	
	
	
	
	
	
	
	
	
	
	
	
	
	%-------------------------------------------------------------------------------------------------------
	\subsection{Research Questions} \label{subsection:research% Formulate specific research questions that your thesis aims to answer.
		TODOquestions}
	
	
	This thesis revolves around the implementation of video colour grading with RGB using Just-In-Time (JIT) techniques. 	
	JIT is used for on-the-fly conversion of video files and optimizing them for streaming. 
	%This process includes the transformation of a video file, potentially one with non-web-friendly formats, into a more suitable format for the playback in web browsers.
	The demand for efficient and dynamic video processing tools is increasing and with this, the ability to perform on-the-fly colour correction becomes increasingly valuable. 
	
	
	The goal of this thesis is to explore the feasibility of implementing the video grading process with JIT, with the focus on real-time colour grading and to implement such a solution, if possible. This project lies in the area of multimedia processing, video editing and real-time computing and is done in cooperation with the company Codemill, that is introduced in Section~\ref{subsection:codemill}.
	
	
	%The infrastructure of the system is visualized and explained in Chapter~\ref{section:designandimplementation}.
	
	The project aims to address the feasibility and effectiveness of implementing video colour correction using JIT in the context of real-time video streaming. The following question is relevant:
	
	\begin{researchbox}
		Is it possible to obtain colour-graded video results on the fly, meaning in real-time, using JIT?
	\end{researchbox}
	
	After answering this research question, the following question will be evaluated and answered:
	
	\begin{researchbox}
		Does the result of the filter application in different applications with a Melt backend and the described system differ? 
	\end{researchbox}

	The Melt framework (MLT/Melt) is a multimedia framework as a command-line (CLI) tool that can be used for video editing and playback. It is introduced and described in Section~\ref{subsection:melt}.

	% Those different applications, that use Melt as a backend are the Melt framework itself and KDEN Live. The Melt framework was introduced and described in Section~\ref{subsection:melt} and KDEN Live in Section~\ref{section:comparisonKDENLive}.

	
	The aim is to contribute insights into the technological and scientific aspects of real-time video processing, exploring the possibilities and potential applications introduced by using JIT to implement colour grading. 
	
	
	
	
	
	
	
	%-------------------------------------------------------------------------------------------------------
	\subsection{Codemill} \label{subsection:codemill}
	% Provide context, highlight the problem space, and explain the motivation behind the project.
	
	Codemill was founded in 2008 in Ume\aa \ (Sweden).~\cite{codemill_now, codemill_old, codemill_linkedin}
	As of February 2024, they employ over 60 employees. \cite{codemill} 
	Codemill is an IT-Consulting company that focussed on the distribution of broadcast media. Their Accurate Video Player that is described in Section~\ref{subsection:accuratevideo}, is part of this thesis and a cloud native software that is being used by the world's leading studios, broadcasters and media service providers.~\cite{codemill_linkedin, codemill_avp}
	The infrastructure of the system that this thesis evolves around is visualized and explained in Chapter~\ref{section:designandimplementation}.
	
	
	
	
	
	
	
	
	
	
	
	
	%-------------------------------------------------------------------------------------------------------
	\subsection{Related Work} \label{subsection:relatedwork}
	
	The topic of this thesis project spans across various fields. 
	Due to this, the related work is dispersed among these fields, highlighting the interdisciplinarity of the topic.
	The fields that are relevant for this project include colour representation, colour grading, multimedia processing, video streaming, the Melt framework, protocol buffers and the comparison of different colour saturations with GIMP.

	
	\begin{description}[font=\color{RedViolet!80!black}, style=nextline]
		
		%---------------------------------------------------------------------------------------------------
		\item[RGB representation] 
		
		One source discussing RGB colour representation and the concepts of additive and subtractive colour models is \textit{Color: An Introduction to Practice and Principles} by Rolf G. Kuehni.~\cite{colourRGB}
		
		
		
		
		%---------------------------------------------------------------------------------------------------
		\item[Colour grading] 
		
		To describe the terminology and differences between colour grading and colour correction according to the industry standards, a variety of web pages and articles were conducted. In this case, web pages were used as sources, to reflect the usage of the terminology in the field. 
		
		One web page is titled \textit{Understanding What Color Correction And Color Grading Are – And Aren’t} from Vegas.~\cite{cc_cg_1} Vegas (former Sony Vegas) is a video editing program from the company Magix, which is widely known as a prominent company within the video editing field. 
		Other web pages were referenced, to confirm the usage of the terms colour grading and colour correction in the industry as described to ensure that the terminology aligns with the commonly accepted standards within the field. This includes the web page with the title \textit{Color Correcting vs. Color Grading: Understanding Film Coloring} from MasterClass, an online learning platform.~\cite{cc_cg_2} 
		
		
		Additionally, the book \textit{The Art and Technique of Digital Color Correction} from Steve Hullfish discusses the topic of colour correction and video post production in more depth.~\cite{cc1}
		
		
		
		%---------------------------------------------------------------------------------------------------
		\item[Effects of different colour]
		
		
		The paper \textit{Color and psychological functioning: a review of theoretical and empirical work} from Andrew J. Elliot discussed theoretical insights and empirical findings on the psychological effect of colours.~\cite{colour}
		
		
		In \textit{Effects of color on emotions} from Patricia Valdez and Albert Mehrabian, emotional reactions to colours were investigated and attributes were assigned to specific colours that they are associated with.~\cite{colour2}
		
		
		
		%---------------------------------------------------------------------------------------------------
		\item[Multimedia processing] 
		
		TODO
		
		
		
		
		
		
		%---------------------------------------------------------------------------------------------------
		\item[Video streaming] 
		
		TODO
		
		
		
		
		
		
		%---------------------------------------------------------------------------------------------------
		\item[Melt framework]
		
		
		The Melt framework is a crucial part for this thesis project. For this, information directly from the Melt website itself was conducted.~\cite{melt} Especially the list of different Melt filters was considered, to compare the Melt filters.~\cite{melt_filters}
		Additionally, the code of the Melt framework was examined.~\cite{melt_code}
		The external literature on the Melt framework is limited.
		
		
		
		
		
		
		
		
		%---------------------------------------------------------------------------------------------------
		\item[Proto Buffer] 
		
		~\cite{protobuffer}
		
		
		
		
		
		
		%---------------------------------------------------------------------------------------------------
		\item[GIMP]
		
		
		~\cite{gimp}
		
		
		
		
		
		
	\end{description}
	
	
	
	
	
	
	
	
	
	\newpage
	%-------------------------------------------------------------------------------------------------------
	\subsection{Structure} \label{subsection:structure}
	% Outline the structure of the thesis, briefly describing each chapter's content.
	
	This thesis consists of seven Chapters. The topics and contents of those Chapters are listed in the following.
	
	
	\begin{itemize}
		
		\item[\textbf{\ref{section:introduction}}] The introduction provides an overview of the thesis, including its motivation, research questions, related work and the structure.
		
		\item[\textbf{\ref{section:theoreticalfoundationofcolour}}] This Chapter contains the theoretical foundation of colour, including the differences between colour correction and grading and the  different effects that different colours can have to a viewer. Furthermore, the representation of colours using RGB and the comparison of different colour saturations is described.
		
		\item[\textbf{\ref{section:technicalbackground}}] In this Chapter the technical background of the project is explained, including an in depth description of the structure and usage of the Melt framework. Additionally, an overview of video streaming components and additional components is given and Protocol Buffers are described.
		
		\item[\textbf{\ref{section:systemrequirementsandspecifications}}] This Chapter outlines the system requirements and specifications for the system to be able to execute the code and it is described how to execute the code.
		
		\item[\textbf{\ref{section:designandimplementation}}] Design and Implementation are described in this Chapter. The architecture of the system is introduced, describing the frontend and backend of the system: Accurate Player and JIT-WebRTC. Furthermore, different Melt filters for RGB adjustment are tested and compared and the implementation of the RGB adjustment is described.
		
		\item[\textbf{\ref{section:experimentalevaluationanddiscussion}}] This Chapter presents the comparison of video colour grading results with Melt filters, that were applied on different platforms. The compared platforms are the Accurate Player RGB adjustment, that was implemented in this thesis project with the locally installed Melt framework and KDEN Live.
		
		\item[\textbf{\ref{section:conclusion}}] The conclusion summarizes the results, contributions and limitations of the thesis project and introduces future work opportunities.
		
	\end{itemize}

	
	
	

	
	

	
	
	
\end{document}