\documentclass[../MasterThesis.tex]{subfiles}
\graphicspath{ {./assets/images/} }


%----------------------------------------------------------------------------
%----------------------------------------------------------------------------

\begin{document}

	%
	%
	%=======================================================================================================
	%
	%
	%
	%
	%=======================================================================================================
	% CHAPTER: INTRODUCTION
	%=======================================================================================================
	\newpage
	\section{Introduction} \label{section:introduction}
	
	
	The popularity of streaming is growing, with an increasing number of people using online streaming platforms for their entertainment and approximately 1.8 billion subscriptions to video streaming services.~\cite{nielsen, stats}
	%
	With this increased popularity, the demand for user-driven real-time visual customization in the web browser for streamed videos has also increased.
	
	
	
	
	
	
	
	
	
	
	
	
	
	
	
	
	
	
	%-------------------------------------------------------------------------------------------------------
	\subsection{Motivation} \label{subsection:motivation}
	
	
	The colouring of a video can evoke different emotions and set an overall tone to the perception of the content. For instance, cool colours such as blue can convey cold weather, while warm colours such as yellow can create a feeling of warmth and sunlight. This can be seen in Figure~\ref{figure:coloursblueandyellow}.
	The colouring of a video is a powerful tool that can significantly impact the conveyed mood and atmosphere.~\cite{colorgrading1, colorgrading2}
	
	% TODO: Only use colograding2 and not 1?
	
	
	\begin{figure}[H]
		\begin{center}
			\cutpic{0.3cm}{0.3\textwidth}{2_blau.jpg}
			%\hspace*{0.01\textwidth}
			\cutpic{0.3cm}{0.3\textwidth}{2.jpg}
			%\hspace*{0.01\textwidth}
			\cutpic{0.3cm}{0.3\textwidth}{2_gelb.jpg}
			\caption[Photo in three different colour tones (original, yellow, blue).]{The original photo can be seen in the middle, while the picture on the left has more blue tones and the picture on the right more yellow tones.}
			\label{figure:coloursblueandyellow}
		\end{center}
	\end{figure}
	%
	%
	%
	The use-cases for on-the-fly adjustments of the video colour arise from different reasons. 
	One of them is that each screen, ranging from computer monitors, TVs, smartphones and tablets to beamers, has unique characteristics that can influence the perceived colour palette. 
	In addition to this, the light conditions can influence the appearance of the colours.
	With real-time colour corrections, the user can adjust the colour and cancel out variations that are caused by the device or environment to have a consistent experience.~\cite{screentype}
	
	Real-time colour grading is a solution that not only addresses the variability in device displays but also enables users to adjust the image according to their preferences.
	%
	Additionally, users with visual problems or specific colour perception might benefit from real-time colour grading to enhance the visibility and distinction of the on-screen elements. 
	The increased accessibility can contribute to enhanced inclusivity and enable that a diverse range of users can engage with the content.~\cite{accessibility}
	
	
	

	
	
	
	
	
	
	
	
	
	
	
	%-------------------------------------------------------------------------------------------------------
	\subsection{Research Questions} \label{subsection:researchquestions}
	
	
	This thesis revolves around the implementation of video colour grading with the adjustment of the RGB values using JIT techniques and the Melt framework. The Melt framework (MLT/Melt) is a multimedia framework as a command-line (CLI) tool that can be used for video editing and playback. It is introduced and described in Section~\ref{subsection:melt}.
	JIT is used for on-the-fly conversion of video files and optimizing them for streaming. 
	%This process includes the transformation of a video file, potentially one with non-web-friendly formats, into a more suitable format for the playback in web browsers.
	The demand for efficient and dynamic video processing tools is increasing and with this, the ability to perform on-the-fly colour correction becomes increasingly valuable. 
	
	
	The goal of this thesis is to explore the feasibility of implementing the video grading process with JIT, with the focus on real-time colour grading and to implement such a solution, if possible. This solution should contain three sliders in the frontend to adjust the individual RGB values: Red, green and blue. The colour adjustment should then be applied to the video output in real-time, while streaming the video.
	This project lies in the area of multimedia processing, video editing and real-time computing and is done in cooperation with the company Codemill, that is introduced in Section~\ref{subsection:codemill}.
	
	
	%The infrastructure of the system is visualized and explained in Chapter~\ref{section:designandimplementation}.
	
	The project aims to address the feasibility and effectiveness of implementing video colour correction using JIT in the context of real-time video streaming. The following question is relevant:
	
	\begin{researchbox}
		Is it possible to obtain colour-graded video results on the fly, meaning in real-time, using JIT and the Melt framework?
	\end{researchbox}
	
	After answering this research question, the following question will be evaluated and answered:
	
	\begin{researchbox}
		Is there variation in the outcomes of filter application across various applications with a Melt backend and the specified system?
	\end{researchbox}


	The different applications for comparison, that use Melt as a backend are the Melt framework itself and the video editing software KDEN Live. 
	
	
	The aim of this thesis project is the contribution of insights into the technological and scientific aspects of real-time video processing, exploring the possibilities and potential applications introduced by using JIT and Melt to implement colour grading. 
	
	
	
	
	
	
\newpage	
	%-------------------------------------------------------------------------------------------------------
	\subsection{Codemill} \label{subsection:codemill}
	% Provide context, highlight the problem space, and explain the motivation behind the project.
	
	Codemill was founded in 2008 in Ume\aa \ (Sweden).~\cite{codemill_now, codemill_old}
	As of February 2024, they employ over 60 employees.~\cite{codemill} 
	Codemill is an IT-Consulting company that focusses on the distribution of broadcast media. Their Accurate Video Player that is described in Section~\ref{subsection:accuratevideo}, is part of this thesis and a cloud native software that is being used by the world's leading studios, broadcasters and media service providers.~\cite{codemill_linkedin, codemill_avp}
	The infrastructure of the system that this thesis evolves around is visualized and explained in Chapter~\ref{section:designandimplementation}.
	
	
	
	
	
	
	
	
	
	
	
	
	%-------------------------------------------------------------------------------------------------------
	\subsection{Related Work} \label{subsection:relatedwork}
	
	The topic of this thesis project spans across various fields. 
	Due to this, the related work is dispersed among these fields, highlighting the interdisciplinarity of the topic.
	The fields that are relevant for this project include colour representation, colour grading, multimedia processing, video streaming, the Melt framework, protocol buffers and the comparison of different colour saturations with GIMP.

	
	\begin{description}[font=\color{RedViolet!80!black}, style=nextline]
		
		%---------------------------------------------------------------------------------------------------
		\item[RGB representation] 
		
		This thesis revolves around RGB colour adjustment, making the understanding of RGB colour representation an important aspect. Additionally, the results of different methods for the RGB colour adjustment are analysed with the \textit{grain extract} function in GIMP, making it necessary to understand the RGB colour model to understand the functionality and results.
		One source discussing RGB colour representation and the concepts of additive and subtractive colour models is \textit{Color: An Introduction to Practice and Principles} by Rolf G. Kuehni.~\cite{colourRGB} 
		
		
		
		
				
		
		%---------------------------------------------------------------------------------------------------
		\item[GIMP]
		
		
		The book \textit{The Book of GIMP: A Complete Guide to Nearly Everything} by Olivier Lecarme and Karine Delvare describes nearly every function in GNU Image Manipulation Program (GIMP).~\cite{gimp} 
		%
		This includes the \textit{grain extract} function, that is relevant for the Melt filter comparison between different platforms in this thesis project.
		%
		Furthermore, understanding techniques for evaluating and editing media content is important within the realm of multimedia processing.
		
		
		
		
		
		%---------------------------------------------------------------------------------------------------
		\item[Colour grading] 
		
		To describe the terminology and differences between colour grading and colour correction according to the industry standards, a variety of web pages and articles were consulted. They are used to reflect the usage of the terminology in the field. 
		% Using the correct terminology ensures effective communication and searchability.		
		%
		One web page is titled \textit{Understanding What Color Correction And Color Grading Are – And Aren’t} from Vegas. Vegas (former Sony Vegas) is a video editing program from the company Magix, which is widely known as a prominent company within the video editing field.~\cite{cc_cg_1}
		%
		Other web pages were referenced to confirm the usage of the terms colour grading and colour correction in the industry as described to ensure that the terminology aligns with the commonly accepted standards within the field. 
		%This includes the web page with the title \textit{Color Correcting vs. Color Grading: Understanding Film Coloring} from MasterClass, an online learning platform.~\cite{cc_cg_2} 
		%
		%
		Furthermore, the book \textit{The Art and Technique of Digital Color Correction} by Steve Hullfish discusses the topic of colour correction and video post-production in more depth.~\cite{cc1}
		
		
		
		%---------------------------------------------------------------------------------------------------
		\item[Effects of different colour]
		
		
		The paper \textit{Color and psychological functioning: a review of theoretical and empirical work} from Andrew J. Elliot discussed theoretical insights and empirical findings on the psychological effect of colours.~\cite{colour}
		%		
		Furthermore, in \textit{Effects of color on emotions} from Patricia Valdez and Albert Mehrabian, emotional reactions to colours were investigated and attributes were assigned to specific colours that they are associated with.~\cite{colour2}
		%
		Understanding the effects of different colours is important for understanding the importance and effects of RGB adjustment and meeting the users needs and requirements, which is relevant for the implementation in this thesis project.
		
	
		
		
		%---------------------------------------------------------------------------------------------------
		\item[Melt framework]
		
		
		The Melt framework is a crucial part for this thesis project. 
		%
		The RGB colour adjustment is implemented using Melt as a backend and the results are compared with other applications using a Melt backend.
		%
		For this, information directly from the Melt website itself was consulted.~\cite{melt} 
		%
		Especially the list of different Melt filters was considered, to compare the Melt filters and select the most suitable option for the RGB adjustment.
		% 
		Additionally, the code of the Melt framework was examined.~\cite{melt_filters, melt_code} 
		%
		The external literature on the Melt framework and its technical aspects is unfortunately limited.
		%
		
		
		
	
		
		
		%---------------------------------------------------------------------------------------------------
		\item[Protocol buffer] 
		
		The Chapter \textit{Proto Buffer} in the book \textit{Mobile Forensics -- The File Format Handbook: Common File Formats and File Systems Used in Mobile Devices} by Chris Currier introduces and discusses the topic of protocol buffers. This is relevant for the data transfer from the frontend to the Melt framework in the implementation for this thesis project.~\cite{protobuffer}
		
		
		
	
		\newpage
		
		%---------------------------------------------------------------------------------------------------
		\item[Multimedia processing] 
		
		The topic of multimedia processing is widely spread and has many different areas. This resulted in many papers, standards and documentations being used as sources to be combined into the overall context of this thesis project. Individual references have been consulted for various components including WebRTC, JIT, video transcoding and other components, which are relevant for this thesis project.
		
		% This includes for example the IEEE standards and publications \textit{IEEE Transactions on Multimedia}, \textit{IEEE Standard for Learning Technology -- JavaScript Object Notation (JSON) Data Model Format and Representational State Transfer (RESTful) Web Service for Learner Experience Data Tracking and Access}, \textit{WebRTC technology overview and signaling solution design and implementation}, \textit{Overview of the H.264/AVC video coding standard} and \textit{Guided just-in-time transcoding for cloud-based video platforms}.~\cite{transcoding, IEEE_Rest, webrtc, h264, JIT_IEEE}
		
		
		
		
		

		
		
		
		
		
	\end{description}
	
	
	
	
	
	
	
	
	
	% \newpage
	%-------------------------------------------------------------------------------------------------------
	\subsection{Structure} \label{subsection:structure}
	% Outline the structure of the thesis, briefly describing each chapter's content.
	
	This thesis consists of seven Chapters. The topics and contents of those Chapters are listed in the following.
	%
	%
	The introduction (Chapter~\ref{section:introduction}) provides an overview of the thesis, including its motivation, research questions, related work and the structure.
	%	
	Chapter~\ref{section:theoreticalfoundationofcolour} contains the theoretical foundation of colour, including the differences between colour correction and grading and the different effects that different colours can have on a viewer. Furthermore, the representation of colours using RGB and the comparison of different colour saturations is described.
	%
	In Chapter~\ref{section:technicalbackground} the technical background of the project is explained, including an in depth description of the structure and usage of the Melt framework. Additionally, an overview of video streaming components and additional components is given and protocol buffers are described.
	%	
	Chapter~\ref{section:systemrequirementsandspecifications} outlines the system requirements and specifications for the system to be able to execute the code. Furthermore it is described how to execute the code.
	%
	The design and implementation of the on-the-fly RGB adjustment are described in Chapter~\ref{section:designandimplementation} . The architecture of the system is introduced, describing the frontend and backend of the system: Accurate Player and JIT-WebRTC. Additionally, different Melt filters for RGB adjustment are tested and compared. At the end of the Chapter the implementation of the RGB adjustment is described.
	%	
	Chapter~\ref{section:experimentalevaluationanddiscussion} presents the comparison of video colour grading results with Melt filters that were applied on different platforms. The compared platforms are the Accurate Player RGB adjustment, that was implemented in this thesis project with the locally installed Melt framework and KDEN Live.
	%	
	The conclusion in Chapter~\ref{section:conclusion} summarizes the results, contributions and limitations of the thesis project and introduces future work opportunities.
		

	
	
	

	
	

	
	
	
\end{document}