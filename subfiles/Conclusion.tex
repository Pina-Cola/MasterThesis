\documentclass[../MasterThesis.tex]{subfiles}
\graphicspath{ {./assets/images/} }


%----------------------------------------------------------------------------
%----------------------------------------------------------------------------

\begin{document}
	
	
%
%
%
%
%=======================================================================================================
%
%
%
%
%=======================================================================================================
% CHAPTER: CONCLUSION AND FUTURE WORK
%=======================================================================================================
\newpage
\section{Conclusion} \label{section:conclusion}


%-------------------------------------------------------------------------------------------------------
\subsection{Summary, Contributions and Limitations} \label{subsection:summary}
% Summarize the key findings and outcomes of the research.




%-------------------------------------------------------------------------------------------------------
\subsection{Future Work} \label{subsection:futurework}
% Suggest possible extensions or improvements to your work.


In this Section, different opportunities for future work are discussed. This includes the implementation of video presets, the application of audio filters, improvements in the user interface and options for advances video processing.


%-------------------------------------------------------------------------------------------------------
\subsubsection*{Video Presets}

Video presets, also known as video filters or video effects, are pre-configured settings or adjustments that can be applied to videos to achieve specific visual styles or effects. 
These presets often include templates for colour grading. 
Video presets are commonly used in video editing software and social media platforms. They provide the user with options to customize their videos easily. 

One example for video presets can be seen in Figure~\ref{figure:app}, where the presets in Adobe Premiere Pro with the \textit{Magic Bullet Looks} plug-in are shown.~\cite{premierepro, magicbullet}

\begin{figure}[H]
	
	\centering
	
	\includegraphics[width=0.99\textwidth]{app.png}
	
	\caption[Presets in Adobe Premiere Pro (\textit{Magic Bullet Looks})]{Presets in Adobe Premiere Pro with the plug-in \textit{Magic Bullet Looks}~\cite{premierepro, magicbullet}}
	\label{figure:app}
	
\end{figure}

Implementing options for the application of those presets is an interesting opportunity for future work. To implement this, different Melt filters can be used or combined. Different options for those Melt filters as video preset filters can be seen in Appendix~\ref{appendix:differentMeltFilter}.












%-------------------------------------------------------------------------------------------------------
\subsubsection*{Improvements of the User Interface}


Improvements of the user interface can be implemented to make the adjustment of the RGB colours more intuitive. This could involve a preview of the colour, that is being created by adjusting the RGB sliders and then applied to the video.
To decide, which changes enhance the usability of the colour adjustment and feel intuitive for the user, extensive testing and user studies could be conducted.


In addition to this, the above described future work opportunity of implementing video presets needs a design for the user interface as well, ro guarantee the usability. 






%-------------------------------------------------------------------------------------------------------
\subsubsection*{Audio filter}

This thesis project focussed on the application of visual filters to a video, especially the adjustment of the RGB values. Aside from this, audio processing plays a role in the processing of videos, too. Similar to the visual content, the audio can enhance the experience and evoke emotions in the viewer. Audio filters can also be used to improve the quality of the recorded audio for example by applying noise suppression or to add context to a scene, by adding fitting surrounding noises. In addition to this, simple aspects including the volume or tone can be modified.
The list of filters on the Melt website contains the available audio filters, too.~\cite{melt}

The implementation of the audio filter integrations is an interesting option for future work. To compare different audio filters, the according sound waves of a track could be read out and compared.












%-------------------------------------------------------------------------------------------------------
\subsubsection*{Advanced Video Processing}

TODO 

\begin{itemize}
	\item Advanced Colour Grading Techniques
	\item Online Video Editing Tool
\end{itemize}








	
	
	
\end{document}