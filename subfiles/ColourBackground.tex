\documentclass[../MasterThesis.tex]{subfiles}
\graphicspath{ {./assets/images/} }


%----------------------------------------------------------------------------
%----------------------------------------------------------------------------

\begin{document}
	

%
%
%
%
%=======================================================================================================
%
%
%
%
%=======================================================================================================
% CHAPTER: Preliminaries
%=======================================================================================================
\newpage
% \section{Preliminaries} \label{section:preliminaries}

\section{Theoretical Foundation of Colour} \label{subsection:theoreticalfoundationofcolour}

In this Chapter, the terms colour correction and colour grading are explained and colour theory is introduced. 
In addition to this, the effects of different colours in media and their association are explained.





%-------------------------------------------------------------------------------------------------------
\subsection{Colour Correction and Colour Grading} 


Colour correction and colour grading are both processes used in film making, photography, and digital art to adjust and enhance the colour of images or videos. 

While they are related, they serve different purposes and are typically performed at different stages of production.

Colour correction aims to correct technical imperfections and achieve a natural-looking image, while colour grading focuses on enhancing the visual aesthetics and storytelling aspects.~\cite{cc_cg_1, cc_cg_2}


Colour correction is the process of adjusting the overall colour balance, contrast, and exposure of an image or video to achieve a more accurate representation of the scene as it was captured by the camera. It involves correcting potential technical issues such as white balance, exposure inconsistencies, and colour inaccuracies caused by the camera or lighting conditions.~\cite{cc1, cc_cg_1, cc_cg_2}

This is typically done in the early stages of post-production to ensure that the footage looks natural and consistent across different shots and scenes.~\cite{cc1}


Colour grading, on the other hand, is the creative process of altering the colour and mood of an image or video to evoke a specific aesthetic or emotional response. The effects of different colours are described in Section~\ref{subsection:effectsofdifferentcolours}.
Colour grading involves making stylistic choices to enhance the visual storytelling or the overall cinematic look of the footage. It can involve manipulating individual colours, adjusting contrast, saturation, and applying various other colour grading techniques.~\cite{cc_cg_1, cc_cg_2}

Colour grading is often performed after colour correction and is considered a more artistic and subjective process. It allows film makers and photographers to establish a unique visual style and enhance the narrative impact of their work.~\cite{cc_cg_1, cc_cg_2}


In summary, colour correction and colour grading serve different purposes and are performed at different stages of post-production. Colour correction ensures technical accuracy and consistency, while colour grading adds artistic flair and enhances visual storytelling.~\cite{cc1, cc_cg_1, cc_cg_2}








%-------------------------------------------------------------------------------------------------------
% \subsection{Colour Theory}










\newpage
%-------------------------------------------------------------------------------------------------------
\subsection{Effects of different colours} 
\label{subsection:effectsofdifferentcolours}

Different colours can convey a different understanding of a scene to the viewer or even influence the viewers emotions. 
The different colours have distinct psychological associations, which influences how viewers perceive and interpret the content.~\cite{colour, colour2}
%But it needs to be considered that colours can carry cultural and contextual meanings that can vary across different societies and individuals. 


In different visual media, including art, photos, or videos, the strategic use of colour can effectively communicate themes, evoke emotions, and guide viewer interpretation.~\cite{cc_cg_1, cc_cg_2, colour, colour2}


In the following, different colours with their psychological associations are listed and example pictures are shown.



%-------------------------------------------------------------------------
%\subsubsection{Red}
\textbf{\textcolor{Maroon}{Red}}

%
%

\begin{minipage}{0.5\textwidth}
	\begin{figure}[H]
		\begin{center}
			\cutpic{0.3cm}{\textwidth}{fliegenpilz.jpg}
			\caption[Picture of a Amanita muscaria in Sweden with alarming red colour.]{Picture of a Amanita muscaria in Sweden with alarming red colour.}
			\label{figure:red}
		\end{center}
	\end{figure}
 \hfill
\end{minipage}\begin{minipage}{0.05\textwidth}
	\ 
\end{minipage}\begin{minipage}{0.45\textwidth}
	The colour red is often associated with intense emotions such as passion, excitement, and danger. Viewing it on self or others has even been shown to increase the perception of aggressiveness and dominance.~\cite{red_dominance} Wearing the colour in sport competitions and events red has been shown to enhance performance and the perceived performance.~\cite{colour, red_sport}
	
	In visual media, the use of red can evoke feelings of urgency, power, and love. For example a video with flashes of red light may create a feeling of tension or alarm while a female wearing red can increase the perceived attraction.~\cite{colour, red_romance}
	
	Beyond its symbolism in human emotions and cultural contexts, red also serves as a prominent warning colour in nature. One example of this is the
	
\end{minipage}

\vspace*{-0.6em}
toxic fly agaric mushroom (Amanita muscaria), which can be seen in Figure~\ref{figure:red}.
The instinctual reaction to red is used in various contexts, including traffic signals and emergency exits, where the color serves as a clear and universally understood signal.







\newpage
%-------------------------------------------------------------------------
%\subsection{Blue} 
\textbf{\textcolor{BlueViolet}{Blue}}

%
\begin{minipage}{0.45\textwidth}
	The colour blue is often linked to feelings of calmness and stability. In videos and pictures, the presence of blue tones can create a sense of relaxation. For instance, the photo in Figure~\ref{figure:blue} of a lake scene in Sweden with shades of blue in the water and the sky can evoke a peaceful mood.
	
	For instance, blue stores and logos have been shown to increase the perception of expected quality and trustworthiness.~\cite{blue_trust, colour2}
	
	But this is not the only association with the colour blue. While it might
	
	
	
\end{minipage}\begin{minipage}{0.05\textwidth}
	\ 
\end{minipage}\begin{minipage}{0.5\textwidth}
	\begin{figure}[H]
		\begin{center}
			\cutpic{0.3cm}{\textwidth}{lake.jpg}
			\caption[Picture of a lake in Sweden with mainly blue tones.]{Picture of a lake in Sweden with mainly blue tones.}
			\label{figure:blue}
		\end{center}
	\end{figure}\hfill
\end{minipage}

\vspace*{-0.6em}
evoke feelings of calmness or trustworthiness in certain contexts, it can also represent sadness or coldness depending on the situation.~\cite{colour2}


%-------------------------------------------------------------------------
%\subsection{Green}
\textbf{\textcolor{ForestGreen}{Green}}

%
%
\begin{minipage}{0.5\textwidth}
	\begin{figure}[H]
		\begin{center}
			\cutpic{0.3cm}{\textwidth}{forest.jpg}
			\caption[Picture of a forest in Sweden with mainly green tones.]{Picture of a forest in Sweden with mainly green tones.}
			\label{figure:green}
		\end{center}
	\end{figure}\hfill
\end{minipage}\begin{minipage}{0.05\textwidth}
	\ 
\end{minipage}\begin{minipage}{0.45\textwidth}
	Green is associated with nature, growth, and harmony. In visual media, the use of green can symbolize renewal, freshness, and balance. Green is a colour that appears often in nature, because it is the result of chlorophyll, a substance in plants that helps them make food from sunlight.~\cite{green, colour2}
	
	For example, the picture in Figure~\ref{figure:green} of a forest in Sweden can evoke feelings of vitality  and  may convey a sense of natural beauty and harmony, because the plants in this picture cause a lot of green tones.
	

	
	
\end{minipage}

	\vspace*{-0.6em}

	Additionally, green is often used in environmental campaigns and eco-friendly branding to signify sustainability and a connection to the earth. 








\newpage
%-------------------------------------------------------------------------
%\subsection{Yellow}
\textbf{\textcolor{YellowOrange}{Yellow}}

%
%
\begin{minipage}{0.45\textwidth}
	Yellow is often associated with happiness, optimism, and energy. When used in videos or pictures, yellow can evoke feelings of warmth, cheerfulness, and positivity. For instance, a photograph capturing a bright yellow sunrise as seen in Figure~\ref{figure:yellow} can convey a sense of warmth, hope and optimism.
	
	In media editing, the colour yellow can be increased to convey the impression sunshine or a sunny day to the viewer.
	
	Moreover, yellow is used in the realm of marketing and advertising. Brands often apply the colour yellow to evoke 
	
	
\end{minipage}\begin{minipage}{0.05\textwidth}
	\ 
\end{minipage}\begin{minipage}{0.5\textwidth}
	\begin{figure}[H]
		\begin{center}
			\cutpic{0.3cm}{\textwidth}{sunset.jpg}
			\caption[Picture of a sunrise in Sweden with mainly yellow tones.]{Picture of a sunrise in Sweden with mainly yellow tones.}
			\label{figure:yellow}
		\end{center}
	\end{figure}\hfill
\end{minipage}

\vspace*{-0.6em}
positive associations with their products, because of its cheerful and optimistic connotations. It also plays a role in interior design: A room painted in soft yellow tones can convey feelings of warmth, making it an ideal choice for homes.~\cite{colour2}




%-------------------------------------------------------------------------
%\subsection{Black and White}

\textbf{\textcolor{gray}{Black and white}}


\begin{minipage}{0.5\textwidth}
	\begin{figure}[H]
	\begin{center}
		\cutpic{0.3cm}{\textwidth}{dubai2.jpg}
		\caption[Picture of a houses in Dubai in greyscales.]{Picture of houses in Dubai in greyscales.}
		\label{figure:gray}
	\end{center}
\end{figure}\hfill
\end{minipage}\begin{minipage}{0.05\textwidth}
	\ 
\end{minipage}\begin{minipage}{0.45\textwidth}
	Black and white or greyscales are often used to evoke a sense of timelessness or simplicity. In visual media, the absence of colour can draw attention to shapes, textures, and contrasts. For example, a black and white photograph of a city skyline can emphasize the architectural details and create a dramatic atmosphere or, depending on the context, it may evoke a sense of nostalgia.
	In Figure~\ref{figure:gray}, houses in Dubai can be seen. The greyscale filter of this photo enhances the focus of the viewer on the architecture and the structures

	
\end{minipage}

\vspace*{-0.6em}
 of the buildings. Furthermore, black, white, and greyscale are often used to convey elegance and simplicity.~\cite{colour2}


%





%-----------------------------------------------------------------------------------
\subsection{RGB Colour Representation}
\label{subsection:RGB}


RGB colour representation is a method to display colours on screens. It stands for Red, Green, and Blue. 
Each of those colour channels (red, green, and blue) has a value ranging from 0 to 255. When all three colours are at their maximum value with \texttt{(255,255,255)}, the outcome is white. Conversely, when all three are at their minimum value with \texttt{(0,0,0}), the result is black.

The three colours are combined in with different values to create the broad spectrum of colours.

For example the colour red can be created with mixing full red (255 in the red channel), no green (0 in the green channel), and no blue (0 in the blue channel). This leads to a pure red as \texttt{(255,0,0)}.
This red can be adjusted by varying the values in the other colour channels. By decreasing the red value, tones such as a dark maroon with \texttt{(130,0,0)} can be achieved and with a higher red and green value a vibrant crimson with \texttt{(200,50,10)} can be achieved. On the other hand, by increasing blue while maintaining a high red channel, tones like scarlet \texttt{(255,40,0)} or coral \texttt{(255,120,80)} can be created. In addition to this, introducing equal amounts of green and blue alongside the high red value can transition the colour into a softer tone, such as rose \texttt{(255,150,150)}. The described colours can be seen in Figure~\ref{figure:RGBred}.




\begin{figure}[H]
	\centering
	
	\begin{tabular}{cccccc}
		
		\textcolor[RGB]{255,0,0}{\rule{2cm}{2cm}} &
		\textcolor[RGB]{130,0,0}{\rule{2cm}{2cm}} &
		\textcolor[RGB]{200,50,10}{\rule{2cm}{2cm}} &
		\textcolor[RGB]{255,40,0}{\rule{2cm}{2cm}} &
		\textcolor[RGB]{255,120,80}{\rule{2cm}{2cm}} &
		\textcolor[RGB]{255,150,150}{\rule{2cm}{2cm}} \\
		
		\scriptsize{\centering \texttt{(255,0,0)}} &
		\scriptsize{\centering \texttt{(130,0,0)}} &
		\scriptsize{\centering \texttt{(200,50,10)}} &
		\scriptsize{\centering \texttt{(255,40,0)}} &
		\scriptsize{\centering \texttt{(255,120,80)}} &
		\scriptsize{\centering \texttt{(255,150,150)}} \\
		
		
		
	\end{tabular}
	
	
	\caption[Different Shades of Red]{Different Shades of Red represented by different RGB values.}
	\label{figure:RGBred}
	
\end{figure}


% RGB(199, 21, 133)

% By adjusting the intensity of each colour channel, it becomes possible to create millions of distinct colours. This technique is widely employed in digital imaging, web design, and various other applications to faithfully portray colours on electronic displays.

% and examples!

% https://www.w3schools.com/colors/colors_rgb.asp

% Poynton, Charles A. "Digital Video and HDTV: Algorithms and Interfaces." Morgan Kaufmann Publishers, 2012.


	
	
	
	
\end{document}