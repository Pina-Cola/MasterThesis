\documentclass[../MasterThesis.tex]{subfiles}
\graphicspath{ {./assets/images/} }


%----------------------------------------------------------------------------
%----------------------------------------------------------------------------

\begin{document}
	

%
%
%
%
%=======================================================================================================
%
%
%
%
%=======================================================================================================
% CHAPTER: Preliminaries
%=======================================================================================================
\newpage
% \section{Preliminaries} \label{section:preliminaries}

\section{Theoretical Foundation of Colour} \label{subsection:theoreticalfoundationofcolour}

In this Chapter, the terms colour correction and colour grading are explained and colour theory is introduced. 
In addition to this, the effects of different colours in media and their association are explained.





%-------------------------------------------------------------------------------------------------------
\subsection{Colour Correction and Colour Grading} 


Colour correction and colour grading are both processes used in film making, photography, and digital art to adjust and enhance the colour of images or videos. 

While they are related, they serve different purposes and are typically performed at different stages of production.

Colour correction aims to correct technical imperfections and achieve a natural-looking image, while colour grading focuses on enhancing the visual aesthetics and storytelling aspects.~\cite{cc_cg_1, cc_cg_2}


Colour correction is the process of adjusting the overall colour balance, contrast, and exposure of an image or video to achieve a more accurate representation of the scene as it was captured by the camera. It involves correcting potential technical issues such as white balance, exposure inconsistencies, and colour inaccuracies caused by the camera or lighting conditions.~\cite{cc1, cc_cg_1, cc_cg_2}

This is typically done in the early stages of post-production to ensure that the footage looks natural and consistent across different shots and scenes.~\cite{cc1}


Colour grading, on the other hand, is the creative process of altering the colour and mood of an image or video to evoke a specific aesthetic or emotional response. The effects of different colours are described in Section~\ref{subsection:effectsofdifferentcolours}.
Colour grading involves making stylistic choices to enhance the visual storytelling or the overall cinematic look of the footage. It can involve manipulating individual colours, adjusting contrast, saturation, and applying various other colour grading techniques.~\cite{cc_cg_1, cc_cg_2}

Colour grading is often performed after colour correction and is considered a more artistic and subjective process. It allows film makers and photographers to establish a unique visual style and enhance the narrative impact of their work.~\cite{cc_cg_1, cc_cg_2}


In summary, colour correction and colour grading serve different purposes and are performed at different stages of post-production. Colour correction ensures technical accuracy and consistency, while colour grading adds artistic flair and enhances visual storytelling.~\cite{cc1, cc_cg_1, cc_cg_2}








%-------------------------------------------------------------------------------------------------------
% \subsection{Colour Theory}










\newpage
%-------------------------------------------------------------------------------------------------------
\subsection{Effects of different colours} 
\label{subsection:effectsofdifferentcolours}

Different colours can convey a different understanding of a scene to the viewer or even influence the viewers emotions. 
The different colours have distinct psychological associations, which influences how viewers perceive and interpret the content.~\cite{colour}
%But it needs to be considered that colours can carry cultural and contextual meanings that can vary across different societies and individuals. 


In different visual media, including art, photos, or videos, the strategic use of colour can effectively communicate themes, evoke emotions, and guide viewer interpretation.~\cite{cc_cg_1, cc_cg_2, colour}


In the following, different colours with their psychological associations are listed and example pictures are shown.



%-------------------------------------------------------------------------
%\subsubsection{Red}
\textbf{\textcolor{Maroon}{Red}}

%
%

\begin{minipage}{0.5\textwidth}
	\begin{figure}[H]
		\begin{center}
			\cutpic{0.3cm}{\textwidth}{fliegenpilz.jpg}
			\label{figure:red}
			\caption[Picture of a Amanita muscaria in Sweden with alarming red colour.]{Picture of a Amanita muscaria in Sweden with alarming red colour.}
		\end{center}
	\end{figure} \hfill
\end{minipage}\begin{minipage}{0.05\textwidth}
	\ 
\end{minipage}\begin{minipage}{0.45\textwidth}
	The colour red is often associated with intense emotions such as passion, excitement, and danger. Viewing it on self or others has even been shown to increase the perception of aggressiveness and dominance.~\cite{red_dominance} Wearing the colour in sport competitions and events red has been shown to enhance performance and the perceived performance.~\cite{colour, red_sport}
	
	In visual media, the use of red can evoke feelings of urgency, power, and love. For example a video with flashes of red light may create a feeling of tension or alarm while a female wearing red can increase the perceived attraction.~\cite{colour, red_romance}
	
	Beyond its symbolism in human emotions and cultural contexts, red also serves as a prominent warning colour in nature. One example of this is the
	
\end{minipage}

\vspace*{-1.2em}
toxic fly agaric mushroom (Amanita muscaria), which can be seen in Figure \ref{figure:red}.
The instinctual reaction to red is used in various contexts, including traffic signals and emergency exits, where the color serves as a clear and universally understood signal.







\newpage
%-------------------------------------------------------------------------
%\subsection{Blue} 
\textbf{\textcolor{BlueViolet}{Blue}}

%
\begin{minipage}{0.45\textwidth}
	The colour blue is often linked to feelings of calmness and stability. In videos and pictures, the presence of blue tones can create a sense of relaxation. For instance, the photo in Figure \ref{figure:blue} of a lake scene in Sweden with shades of blue in the water and the sky can evoke a peaceful mood.
	
	For instance, blue stores and logos have been shown to increase the perception of expected quality and trustworthiness.~\cite{blue_trust}
	
	But this is not the only association with the colour blue. While it might
	
	
	
\end{minipage}\begin{minipage}{0.05\textwidth}
	\ 
\end{minipage}\begin{minipage}{0.5\textwidth}
	\begin{figure}[H]
		\begin{center}
			\cutpic{0.3cm}{\textwidth}{lake.jpg}
			\label{figure:blue}
			\caption[Picture of a lake in Sweden with mainly blue tones.]{Picture of a lake in Sweden with mainly blue tones.}
		\end{center}
	\end{figure}\hfill
\end{minipage}

\vspace*{-1.2em}
evoke feelings of calmness or trustworthiness in certain contexts, it can also represent sadness or coldness depending on the situation.


%-------------------------------------------------------------------------
%\subsection{Green}
\textbf{\textcolor{ForestGreen}{Green}}

%
%
\begin{minipage}{0.5\textwidth}
	\begin{figure}[H]
		\begin{center}
			\cutpic{0.3cm}{\textwidth}{forest.jpg}
			\label{figure:green}
			\caption[Picture of a forest in Sweden with mainly green tones.]{Picture of a forest in Sweden with mainly green tones.}
		\end{center}
	\end{figure}\hfill
\end{minipage}\begin{minipage}{0.05\textwidth}
	\ 
\end{minipage}\begin{minipage}{0.45\textwidth}
	Green is associated with nature, growth, and harmony. In visual media, the use of green can symbolize renewal, freshness, and balance. Green is a colour that appears often in nature, because it is the result of chlorophyll, a substance in plants that helps them make food from sunlight.
	
	For example, the picture in Figure \ref{figure:green} of a forest in Sweden can evoke feelings of vitality  and  may convey a sense of natural beauty and harmony, because the plants in this picture cause a lot of green tones.
	
	Additionally, green is often used in environmental campaigns and eco-friendly branding to signify sustainability and a connection to the earth. 
	
	
	
	% Moreover, the prevalence of green extends beyond terrestrial landscapes to aquatic realms, where algae and seaweeds paint the ocean depths with verdant hues. From the dense canopy of rainforests to the rolling meadows of the countryside, green pervades every corner of our planet, serving as a visual testament to the vitality and resilience of life itself.
	
	% In this context, the color green emerges as more than just a pigment; it is a living embodiment of the interconnectedness between humanity and the natural world. As stewards of the Earth, we are called upon to cherish and protect the green sanctuaries that sustain life in all its forms.
	
	
	
	% In the realm of visual communication, the strategic use of green transcends mere aesthetics. It serves as a potent symbol of renewal, reflecting the perpetual cycle of life found in nature. Artists and designers harness its vibrant hues to evoke feelings of vitality and rejuvenation. Take, for instance, the captivating image captured in Figure \ref{figure:green}, showcasing a verdant forest in Sweden. Here, the lush greenery not only mesmerizes the viewer but also instills a profound sense of connection to the natural world. It speaks to the enduring beauty and harmony inherent in the wilderness.
	
	% Furthermore, green's association with eco-friendliness and sustainability has made it a cornerstone in environmental campaigns and branding initiatives. Companies seeking to align themselves with ecological values often integrate shades of green into their logos and packaging. This deliberate choice not only signals a commitment to environmental stewardship but also resonates with consumers who prioritize ethical consumption. Brands like Patagonia and The Body Shop exemplify this ethos, leveraging green imagery to foster trust and loyalty among environmentally conscious consumers.
	
	% Delving deeper, the significance of green extends beyond its visual appeal. It embodies a profound connection to the Earth and its ecosystems, reminding us of our interdependence with the natural world. Studies in environmental psychology have shown that exposure to green spaces can have therapeutic effects, reducing stress and enhancing overall well-being. In urban environments, the incorporation of parks and gardens serves as vital sanctuaries, offering respite from the concrete jungle.
	
	% Moreover, green's symbolic potency is deeply rooted in cultural traditions and spiritual beliefs. Across various civilizations, it has been revered as a symbol of fertility, abundance, and rebirth. In ancient Egyptian mythology, Osiris, the god of the afterlife, was often depicted with green skin, symbolizing resurrection and eternal life. Similarly, in Eastern philosophies such as Feng Shui, green is associated with the Wood element, representing growth, vitality, and prosperity.
	
	% In contemporary society, the allure of green continues to captivate hearts and minds alike. From sustainable architecture to eco-friendly fashion, its influence permeates every facet of modern living. As we navigate an era defined by environmental challenges, the symbolism of green serves as a poignant reminder of our responsibility to protect and preserve the planet for future generations.
	
	% In conclusion, the multifaceted symbolism of green transcends its role as a mere color. It serves as a potent symbol of vitality, renewal, and ecological consciousness. Whether adorning the canvas of a masterpiece or gracing the packaging of a sustainable product, green remains a timeless emblem of our enduring connection to the natural world.
	
	% References:
	
	%Kaplan, R., & Kaplan, S. (1989). The Experience of Nature: A Psychological Perspective. Cambridge University Press.
	%Korpela, K., Ylén, M., Tyrväinen, L., & Silvennoinen, H. (2010). Determinants of restorative experiences in everyday favorite places. Health & Place, 16(3), 457-465.
	%Lipovetsky, S. (2013). Green: A Field Guide to Marijuana. Chronicle Books.
	%Moore, E. O. (1981). A prison environment's effect on health care service demands. Journal of Environmental Systems, 11(1), 17-34.
	%Wurtman, R. J. (1975). The Effects of Light on the Human Body. Scientific American, 233(4), 68-77.
	
	
\end{minipage}


\newpage
%-------------------------------------------------------------------------
%\subsection{Yellow}
\textbf{\textcolor{YellowOrange}{Yellow}}

%
%
\begin{minipage}{0.45\textwidth}
	Yellow is often associated with happiness, optimism, and energy. When used in videos or pictures, yellow can evoke feelings of warmth, cheerfulness, and positivity. For instance, a photograph capturing a bright yellow sunrise as seen in Figure \ref{figure:yellow} can convey a sense of warmth, hope and optimism.
	
	TODO references
	
	TODO more text
	
	TODO more text
	
\end{minipage}\begin{minipage}{0.05\textwidth}
	\ 
\end{minipage}\begin{minipage}{0.5\textwidth}
	\begin{figure}[H]
		\begin{center}
			\cutpic{0.3cm}{\textwidth}{sunset.jpg}
			\label{figure:yellow}
			\caption[Picture of a sunrise in Sweden with mainly yellow tones.]{Picture of a sunrise in Sweden with mainly yellow tones.}
		\end{center}
	\end{figure}\hfill
\end{minipage}







%-------------------------------------------------------------------------
%\subsection{Black and White}

\textbf{\textcolor{gray}{Black and white}}


\begin{minipage}{0.5\textwidth}
	\begin{figure}[H]
	\begin{center}
		\cutpic{0.3cm}{\textwidth}{dubai2.jpg}
		\label{figure:gray}
		\caption[Picture of a houses in Dubai in greyscales.]{Picture of houses in Dubai in greyscales.}
	\end{center}
\end{figure}\hfill
\end{minipage}\begin{minipage}{0.05\textwidth}
	\ 
\end{minipage}\begin{minipage}{0.45\textwidth}
	Black and white or greyscales are often used to evoke a sense of timelessness or simplicity. In visual media, the absence of colour can draw attention to shapes, textures, and contrasts. For example, a black and white photograph of a city skyline can emphasize the architectural details and create a dramatic atmosphere or, depending on the context, it may evoke a sense of nostalgia.
	
	TODO more text
	
	TODO more text
	
\end{minipage}

%

%\section{Conclusion}

%The emotional impact of colors in videos and pictures cannot be overstated. By understanding the psychological associations of different colors, creators can effectively convey emotions, set the mood, and engage viewers on a deeper level. Whether it's the passionate reds, calming blues, or vibrant yellows, each color has the power to evoke a unique emotional response, enriching the visual experience for the audience.


%Picture ideas:    

%Red: A picture of a sunset with rich red hues, or a close-up shot of a red rose.

%Blue: Capture an image of a tranquil lake with blue reflections, or a serene sky with fluffy clouds.

%Green: Take a photograph of a lush forest with vibrant green foliage, or a close-up of fresh grass after rainfall.

%Yellow: Capture the golden glow of a sunrise or sunset, or photograph a field of sunflowers bathed in sunlight.

%Black and White: Photograph a classic cityscape with stark contrasts between buildings and shadows, or capture the simplicity of a monochrome portrait.





	
	
	
	
\end{document}