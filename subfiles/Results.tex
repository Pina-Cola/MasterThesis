\documentclass[../MasterThesis.tex]{subfiles}
\graphicspath{ {./assets/images/} }


%----------------------------------------------------------------------------
%----------------------------------------------------------------------------

\begin{document}
	
	
%
%
%
%
%=======================================================================================================
%
%
%
%
%=======================================================================================================
% CHAPTER: RESULTS AND EVALUATION
%=======================================================================================================
\newpage
\section{Comparison of Video Colour Grading Results} \label{section:experimentalevaluationanddiscussion}  
% \section{Experimental Evaluation and Discussion} \label{section:experimentalevaluationanddiscussion}


In this Chapter, the results of the user-sided colour grading with JIT are compared to the results of the colour grading with the Melt framework CLI tool and KDEN Live, which uses the Melt framework as a backend. For the comparison, frames of the original video file, the frames created by the JIT implementation and the Melt framework and KDEN Live are compared with and without the application of filters.


To evaluate the difference in the frames and the filter application between the tested components, the frames are extracted. Those tests were run on the same computer to ensure consistency and the frames were extracted by taking screenshots of the output or pre-existing frame extraction functions.
Then, the colours of the images have been subtracted from each other using the \textit{grain extract} function in GIMP, as described in Section~\ref{subsection:ComparisonOfDifferentColourSaturations}.


In the Figures~\ref{figure:greyresult_unit} and \ref{figure:greyresult_blurry}, two different results of the colour subtraction can be seen. 
%
Figure~\ref{figure:greyresult_unit} shows the subtraction of an image from itself. This results in a uniformly grey image, because the images are identical. Figure~\ref{figure:greyresult_blurry} in comparison shows the subtractions of two separate screenshots of the same frame. Small distortions can be seen, which is caused by slightly different scaling or other inconsistencies in the screenshot process. Those small differences have to be considered in the comparison, because the images can not be expected to be exactly the same, when extracting them from different sources. Two subtracted frames that have a similar result to Figure~\ref{figure:greyresult_blurry} can be considered as the same frames.


\begin{minipage}{0.48\textwidth}
	
	\begin{figure}[H]
		\begin{center}
			\cutpic{0cm}{0.99\textwidth}{sameVSsame.png}
			\caption[Colour subtraction of the same file from itself.]{Colour subtraction of the same file from itself results in a uniformly grey image.}
			\label{figure:greyresult_unit}
		\end{center}
	\end{figure}
\end{minipage}\begin{minipage}{0.04\textwidth}
	\ 
\end{minipage}\begin{minipage}{0.48\textwidth}	
	\begin{figure}[H]
		\begin{center}
			\cutpic{0cm}{0.99\textwidth}{simVSsim.png}
			\caption[Colour subtraction of the same frame but a different screenshot.]{Colour subtraction of the same frame but a different screenshot, which results in slight distortions.}
			\label{figure:greyresult_blurry}
		\end{center}
	\end{figure}
\end{minipage}







%-------------------------------------------------------------------------------------------------------
%-------------------------------------------------------------------------------------------------------


\subsection{Comparison with the Melt Framework} \label{section:comparisonMelt}


In this Section, the results of video colour grading using the locally installed Melt framework and the in Section~\ref{subsection:implementation} described implementation with JIT are compared. 
Those tests are run on the same computer to ensure consistency and the frames are extracted with taking screenshots of the output from both approaches. These extracted frames are then visually inspected. 


For the evaluation, a comparison between the locally installed Melt version with the \texttt{avfilter.colorbalance} filter applied and the RGB adjustment in the Accurate Player with the sliders is made.
For the tests, the red value is set to the maximum with the respective method and the results will be compared and analysed below. The results and comparisons of the colours green and blue perform analogical, which can be seen in Appendix~\ref{appendix:comparisonRGB}.
%, assuming that the results of the green and blue value work analogical.
The filter is applied and frame 343 of the short film \textit{Sintel} is extracted from both components. The results of the locally installed Melt framework can be seen in Figure~\ref{figure:redMelt}, with the \texttt{avfilter.colorbalance} filter applied and the parameters for shadows, mid tones and highlights set to the maximum of $1$ (\texttt{av.rs=1}, \texttt{av.rm=1} and \texttt{av.rh=1}).
In comparison, the extracted frame from the Accurate Player frontend can be seen in Figure~\ref{figure:redAP}. This result is obtained by setting the red slider in the frontend onto the maximum value and leaving the sliders for green and blue on the default value, which corresponds to $0$ in the filter application in the backend. The maximum value of the red slider gets sent to the backend and is being applied for the parameters of the filter \texttt{avfilter.colorbalance}, setting the parameters for shadows, mid tones and highlights to the maximum of $1$ (\texttt{av.rs=1}, \texttt{av.rm=1} and \texttt{av.rh=1}) as well.



\begin{minipage}{0.48\textwidth}
	
	\begin{figure}[H]
		\begin{center}
			\cutpic{0.3cm}{0.99\textwidth}{red_Melt.png}
			\caption[Frame 343 after the application of the red filter in the Melt framework.]{Frame 343 after the application of the maximum value for red in the Melt framework.}
			\label{figure:redMelt}
		\end{center}
	\end{figure}
\end{minipage}\begin{minipage}{0.04\textwidth}
	\ 
\end{minipage}\begin{minipage}{0.48\textwidth}
	
	\begin{figure}[H]
		\begin{center}
			\cutpic{0.3cm}{0.99\textwidth}{red_AP.png}
			\caption[Frame 343 after the application of the red filter in the Accurate Player.]{Frame 343 after the application of the maximum value for red in the Accurate Player.}
			\label{figure:redAP}
		\end{center}
	\end{figure}
\end{minipage}

\vspace*{1.5em}
In these Figures it can be seen that the red tone of the two frames visually differs. In Figure~\ref{figure:redAP}, the colours seem to be darker or more saturated. This is an unexpected result, because both components use the same Melt filter with the same parameters.


Considering the visual difference in the frame extractions from the two different approaches of applying the filter \texttt{avfilter.colorbalance}, the subtraction of those two frames shows more differences than the expected distortions, that were described above.

In Figure~\ref{figure:filterVSfilter}, the extracted frame of the Melt framework is subtracted from the frame of the Accurate Player.
In this case, the result has a red tone itself, because the seemingly less red saturated frame (Figure~\ref{figure:redMelt}) is subtracted from the seemingly more red saturated frame (Figure~\ref{figure:redAP}). The pixels in the Accurate Player frame have higher RGB values on the red colour channel than the corresponding pixels in the Melt framework extraction. This causes the red tone in the resulting image in Figure~\ref{figure:filterVSfilter}.
This shows the relative differences in the red value between the two frames. 
The result image from the subtraction is red tinted and has no green or blue areas, which reveals that the frame of the Accurate Player has overall a more saturated red tone than the extracted frame of the Melt framework.


\begin{figure}[H]
	\begin{center}
		\cutpic{0.3cm}{0.65\textwidth}{filterVSfilter.png}
		\caption[Subtraction of the two different frames (Accurate Player - Melt).]{Subtraction of the same frame with filter application: Local Melt frame is subtracted from the frame of the Accurate Player.}
		% JIT - Melt
		\label{figure:filterVSfilter}
	\end{center}
\end{figure}

To evaluate, if this difference is caused by the application of the filter or by other aspects, such as video compression or difference in video players, the frames from the original video file of the short film \textit{Sintel} are compared with the frames from both methods without a filter applied to them. For this, frame 343 of the original video file is compared to the same frame extraction of the locally installed Melt framework and the Accurate Player without the application of the filter. The frame from the original video file can be seen in Figure~\ref{figure:nofilterO}.

\begin{figure}[H]
	\begin{center}
		\cutpic{0.3cm}{0.65\textwidth}{nofilter_org.png}
		\caption[Frame 343 without filter application in the original video file.]{Frame 343 without filter application in the original video file.}
		\label{figure:nofilterO}
	\end{center}
\end{figure}


In Figure~\ref{figure:nofilterMelt}, the extraction of frame 343 without a filter application from the locally installed Melt framework can be seen. In comparison to this, the extracted frame from the Accurate Player without an applied filter is shown in Figure~\ref{figure:nofilterAP}, which seems to be more saturated.


\begin{minipage}{0.48\textwidth}
	\begin{figure}[H]
		\begin{center}
			\cutpic{0.3cm}{0.99\textwidth}{nofilter_melt.png}
			\caption[Frame 343 without filter application in the Melt framework.]{Frame 343 without filter application in the Melt framework.}
			\label{figure:nofilterMelt}
		\end{center}
	\end{figure}
\end{minipage}\begin{minipage}{0.04\textwidth}
	\ 
\end{minipage}\begin{minipage}{0.48\textwidth}
	\begin{figure}[H]
		\begin{center}
			\cutpic{0.3cm}{0.99\textwidth}{nofilter_ap.png}
			\caption[Frame 343 without filter application in the Accurate Player.]{Frame 343 without filter application in the Accurate Player.}
			\label{figure:nofilterAP}
		\end{center}
	\end{figure}
\end{minipage}

\vspace*{1.5em}
To analyse the differences between the frames without a filter of the two methods further, another frame subtraction is conducted. Each of the frames are compared to the extraction of the original video file, to identify, which method is altering the results.
In Figure~\ref{figure:oVSmelt}, the subtraction of the Melt Framework frame from the original frame is visualized using the \textit{grain extract} function. In this image, the outlines of the motive of the frame can be seen in different grey tones, which reveals, that the subtracted frames do not have the same colour values.
In Figure~\ref{figure:oVSap} in comparison, the subtraction is done with the Accurate Player frame from the original frame. Figure~\ref{figure:oVSap} shows the expected result for a same output, with some differences in the grey tones but no clear outlines or fragments from the original frame motive.



\begin{minipage}{0.48\textwidth}
	\begin{figure}[H]
		\begin{center}
			\cutpic{0.3cm}{0.99\textwidth}{oVSmelt.png}
			\caption[Subtraction of the Melt frame from the original frame.]{Subtraction of the Melt frame from the original frame.}
			% O - Melt
			\label{figure:oVSmelt}
		\end{center}
	\end{figure}
\end{minipage}\begin{minipage}{0.04\textwidth}
	\ 
\end{minipage}\begin{minipage}{0.48\textwidth}
	\begin{figure}[H]
		\begin{center}
			\cutpic{0.3cm}{0.99\textwidth}{oVSap.png}
			\caption[Subtraction of Accurate Player frame from the original frame.]{Subtraction of Accurate Player frame from the original frame.}
			% O - JIT
			\label{figure:oVSap}
		\end{center}
	\end{figure}
\end{minipage}





% One potential factor contributing to these discrepancies could be variations in the video player used within the Melt framework, although this remains uncertain. 



\vspace*{1.5em}
The extracted frames from the different approaches display significant differences, even when no filter is applied. 
The results show, that the Accurate Player plays back the video more accurate than the Melt framework.
These differences in the video representation without a filter application make it difficult to evaluate additional data regarding the application of filters using this method. The cause for the different display of the video frames by the Melt framework remains unclear.
% It is worth noting that the Melt video player is recognized for encountering occasional technical issues.
%
%
One possible explanation for these differences in the extracted frames could be rooted in the system environment potentially influencing the evaluation of the comparison on different platforms. The differences could be caused by software dependencies imposed by the Melt framework or its video player.













%-------------------------------------------------------------------------------------------------------
%-------------------------------------------------------------------------------------------------------


\subsection{Comparison with KDEN Live} \label{section:comparisonKDENLive}


% \vspace*{-1em}
\begin{minipage}{0.58\textwidth}
	
	In this Section, the results of video colour grading using KDEN Live and the in Section~\ref{subsection:implementation} described implementation with JIT are compared. In Figure~\ref{figure:kdenlive_effects}, the list of categories for the available effects in KDEN Live can be seen. This includes the \textit{Color and Image correction} option, which will be examined in the following. The options for the RGB colour adjustment in KDEN Live are examined and the results are compared to the frames of the Accurate Player. With those results it can be seen, if KDEN Live uses the \texttt{avfilter.colorbalance} filter or a different 
	
	
\end{minipage}\begin{minipage}{0.04\textwidth}
	\ 
\end{minipage}\begin{minipage}{0.38\textwidth}
	\begin{figure}[H]
		\begin{center}
			\cutpic{0.3cm}{0.9\textwidth}{kdenlive_effects.png}
			\caption[List of effect menus in KDEN Live.]{List of effect menus in KDEN Live.}
			\label{figure:kdenlive_effects}
		\end{center}
	\end{figure}
	\vfill
\end{minipage}

\vspace*{0.2em}

filter for the adjustment of the RGB values. 
The comparison of different filters is a difficult task, because it cannot be evaluated, which aspects of the results differ due to the different filters and which aspects are influenced by other aspects.


In the effect category \textit{Color and Image correction} in KDEN Live, different effects for colours can be found. The list of effects in this category can be seen in Figure~\ref{figure:kdenlive_effekte}. 


% \vspace*{-0.3em}
\begin{minipage}{0.48\textwidth}
	\vspace*{-0.3em}
	\begin{figure}[H]
	\begin{center}
		\cutpic{0.3cm}{0.9\textwidth}{kdenlive_effekte.png}
		\caption[List of colour effects in KDEN Live.]{List of colour effects (KDEN Live).}
		\label{figure:kdenlive_effekte}
	\end{center}
	\end{figure}
	\vfill
	
	
\end{minipage}\begin{minipage}{0.04\textwidth}
	\ 
\end{minipage}\begin{minipage}{0.48\textwidth}
%
%
Those effect options were examined and \textit{RGB adjustment} is the only effect, that is adjusting the colour of the produced video with RGB values. This adjustment is implemented with sliders for red, green and blue, similar to the user-sided RGB adjustment in the Accurate Player using JIT. The other effects in the list, that can be seen in Figure~\ref{figure:kdenlive_effekte}, are not adjusting the RGB colouring of the video.
For example the effect \textit{3 point balance} might sound promising as well but refers to the adjustment of the values for black, gray and white.

\end{minipage}

\vspace*{0.6em}

In Figure~\ref{figure:kdenlive_rgb}, the RGB colour sliders of the effect \textit{RGB adjustment} can be seen. As additional parameter, the following different actions can be chosen: Change gamma, add constant and multiply. For each of those actions a comparison through colour subtraction with the \textit{grain extract} function in GIMP is performed. Those frames from KDEN Live are compared with the extracted frames from the Accurate Player, with the application of the filter \texttt{avfilter.colorbalance} in the backend.
The frames were extracted from the Accurate Player through taking screenshots of the output and in KDEN Live the \textit{extract frame} function was used.


\begin{figure}[H]
	\begin{center}
		\cutpic{0cm}{0.5\textwidth}{kdenlive_rgb.png}
		\caption[RGB colour adjustment in KDEN Live.]{RGB colour adjustment in KDEN Live.}
		\label{figure:kdenlive_rgb}
	\end{center}
\end{figure}

For the other additional parameters (keep luma, alpha controlled and luma formula), the default values were used for this comparison.












%-------------------------------------------------------------------------------------------------------
\subsubsection*{Change gamma}

\textit{Change gamma} refers to adjusting the luminance or brightness of an image or video by applying a gamma correction factor. This can be used to adapt the output to different devices.~\cite{gamma}

In Figure~\ref{figure:gamma}, the extracted frame from KDEN Live with the application of the maximum red value with the \textit{Change gamma} action is shown. The slider for the adjustment of the colour red is set to the maximum while the other sliders remain unchanged.
In comparison, the extracted frame from the Accurate Player with the maximum red value applied can be seen in Figure~\ref{figure:APframe1}. This result is obtained by setting the red slider in the frontend onto the maximum value and leaving the sliders for green and blue on the default value. In the backend, the filter \texttt{avfilter.colorbalance} with the parameters for red shadows, mid tones and highlights to the maximum of $1$ is applied (\texttt{av.rs=1}, \texttt{av.rm=1} and \texttt{av.rh=1}).
This frame from the Accurate Player was shown before in Figure~\ref{figure:redAP} and is displayed here again for a better visualization of the frame comparison.

The frame extraction from KDEN Live (Figure~\ref{figure:gamma}) is visually darker than the frame extraction from the Accurate Player (Figure~\ref{figure:APframe1}). This leads to the assumption, that the KDEN Live effect \textit{RGB adjusment} is not based on the functionalities of the Melt filter \texttt{avfilter.colorbalance}.

\begin{minipage}{0.48\textwidth}
	\begin{figure}[H]
		\begin{center}
			\cutpic{0.3cm}{0.99\textwidth}{red_AP.png}
			\caption[Frame 343 after the application of the red filter in the Accurate Player.]{Frame 343 after the application of the maximum value for red in the Accurate Player.}
			\label{figure:APframe1}
		\end{center}
	\end{figure}
\end{minipage}\begin{minipage}{0.04\textwidth}
	\ 
\end{minipage}\begin{minipage}{0.48\textwidth}
	\begin{figure}[H]
		\begin{center}
			\cutpic{0.3cm}{0.99\textwidth}{gamma.png}
			\caption[Frame 343 from KDEN Live with the action \textit{change gamma}.]{Frame 343 from KDEN Live with the action \textit{change gamma} in the red filter application.}
			\label{figure:gamma}
		\end{center}
	\end{figure}
\end{minipage}

\vspace*{1.2em}
% \newpage

The result of the subtraction can be seen Figure~\ref{figure:gammagimp}. The Accurate Player frame was subtracted from the KDEN Live frame. The result does show green tones and no red tones, which demonstrates that the Accurate Player frame has more saturated red values than the KDEN Live frame in all areas. Because the frames differ, the outlines and fragments of the original motive of the frame can be seen.



\begin{figure}[H]
	\begin{center}
		\cutpic{0.3cm}{0.65\textwidth}{gamma_gimp.png}
		\caption[Subtraction of KDEN Live (\textit{change gamma}) and Accurate Player.]{Subtraction of the two frames: Accurate Player frame is subtracted from the KDEN Live frame (with the action \textit{change gamma}).}
		% KDENLive - JIT
		\label{figure:gammagimp}
	\end{center}
\end{figure}








%-------------------------------------------------------------------------------------------------------
\subsubsection*{Add constant}

Adding a constant value involves increasing or decreasing the intensity of a colour channel in an image or video by a fixed amount. Assuming, the Melt filter \texttt{avfilter.colorbalance} adjusts the RGB colour channels by a fixed amount, which is set by the filter parameters, this action could lead to more similar results to the Accurate Player frame.

In Figure~\ref{figure:addconstant}, the extracted frame from KDEN Live with the application of the maximum red value for the \textit{Add constant} action is displayed. 
In Figure~\ref{figure:APframe2} in comparison, the extracted frame from the Accurate Player with the maximum red value applied can be seen. This frame was shown before in Figure~\ref{figure:redAP} and \ref{figure:APframe1} and is displayed here again to allow for a direct comparison. The adjustment of the red slider to the maximum value in the frontend leads to the application of the filter \texttt{avfilter.colorbalance} with the red parameters set to the maximum value of $1$ in the backend (\texttt{av.rs=1}, \texttt{av.rm=1} and \texttt{av.rh=1}).

The KDEN Live frame extraction (Figure~\ref{figure:addconstant}) is visually darker than the frame extraction from the Accurate Player (Figure~\ref{figure:APframe2}) and looks similar to the frame that was created with the \textit{Change gamma} action that was shown in Figure~\ref{figure:gamma}.

\begin{minipage}{0.48\textwidth}
	\begin{figure}[H]
		\begin{center}
			\cutpic{0.3cm}{0.99\textwidth}{red_AP.png}
			\caption[Frame 343 after the application of the red filter in the Accurate Player.]{Frame 343 after the application of the maximum value for red in the Accurate Player.}
			\label{figure:APframe2}
		\end{center}
	\end{figure}
\end{minipage}\begin{minipage}{0.04\textwidth}
	\ 
\end{minipage}\begin{minipage}{0.48\textwidth}
	\begin{figure}[H]
		\begin{center}
			\cutpic{0.3cm}{0.99\textwidth}{addconstant.png}
			\caption[Frame 343 from KDEN Live with the action \textit{add constant}.]{Frame 343 from KDEN Live with the action \textit{add constant} in the red filter application.}
			\label{figure:addconstant}
		\end{center}
	\end{figure}
\end{minipage}
\vspace*{1em}


The subtraction of those frames can be seen in Figure~\ref{figure:addconstantgimp}. It reveals, that the images that were created with the \textit{Change gamma} and the \textit{Add constant} action differ more than visually noticeable at first glance. The result of the subtraction of the Accurate Player frame from the KDEN Live frame extraction in Figure~\ref{figure:addconstantgimp} shows, that the Accurate Player frame has a more saturated red value in some areas, while in other areas the KDEN Live frame extraction has a more saturated red value. The resulting image contains red and green pixels after applying the \textit{grain extract} function for the subtraction in GIMP.


\begin{figure}[H]
	\begin{center}
		\cutpic{0.3cm}{0.65\textwidth}{addconstant_gimp.png}
		\caption[Subtraction of KDEN Live (\textit{add constant}) and Accurate Player.]{Subtraction of the two frames: Accurate Player frame is subtracted from the KDEN Live frame extraction (with the action \textit{add constant}).}
		% KDENLive - JIT
		\label{figure:addconstantgimp}
	\end{center}
\end{figure}







%-------------------------------------------------------------------------------------------------------
\subsubsection*{Multiply}

Multiplying involves changing the intensity of each colour channel (red, green, blue) by a specified factor. This adjustment allows for precise control over colour balance and saturation.~\cite{gimp}

In Figure~\ref{figure:multiply}, the extracted frame from KDEN Live with the application of the maximum red value for the \textit{Multiply} action is displayed. 
In comparison in Figure~\ref{figure:APframe2}, the extracted frame from the Accurate Player with the maximum red value applied can be seen. This frame was shown and explained before and is displayed here again to allow for a direct visual comparison of the frames.

The frame extraction from KDEN Live (Figure~\ref{figure:multiply}) is visually more saturated and red toned than the frame extraction from the Accurate Player (Figure~\ref{figure:APframe2}).


\begin{minipage}{0.48\textwidth}
	\begin{figure}[H]
		\begin{center}
			\cutpic{0.3cm}{0.99\textwidth}{red_AP.png}
			\caption[Frame 343 after the application of the red filter in the Accurate Player.]{Frame 343 after the application of the maximum value for red in the Accurate Player.}
			\label{figure:APframe3}
		\end{center}
	\end{figure}
\end{minipage}\begin{minipage}{0.04\textwidth}
	\ 
\end{minipage}\begin{minipage}{0.48\textwidth}
	\begin{figure}[H]
		\begin{center}
			\cutpic{0.3cm}{0.99\textwidth}{multiply.png}
			\caption[Frame 343 from KDEN Live with the action \textit{Multiply}.]{Frame 343 from KDEN Live with the action \textit{Multiply} in the red filter application.}
			\label{figure:multiply}
		\end{center}
	\end{figure}
\end{minipage}
\vspace*{1em}

In Figure~\ref{figure:multiplygimp}, the subtraction of the Accurate Player frame extration from the KDEN Live frame can be seen. This subtraction result shows a mostly red picture, caused by the high red values in the KDEN Live frame. In the green areas on the other hand, the Accurate Player frame had a more saturated red value than the KDEN Live frame. This is presumably caused by the darkness of those areas in the frame that was extracted from KDEN Live.



\begin{figure}[H]
	\begin{center}
		\cutpic{0.3cm}{0.65\textwidth}{multiply_gimp.png}
		\caption[Subtraction of KDEN Live (\textit{Multiply} and Accurate Player).]{Subtraction of the two frames: Accurate Player frame is subtracted from the KDEN Live frame extraction (with the action \textit{Multiply}).}
		% KDEN Live - JIT
		\label{figure:multiplygimp}
	\end{center}
\end{figure}












%-------------------------------------------------------------------------------------------------------
%-------------------------------------------------------------------------------------------------------


\subsection{Conclusion of the Filter Comparison} \label{section:resultscomparisons}


In this Chapter, the results of the user-sided colour grading with JIT were compared to the results of the colour grading with the Melt framework CLI tool and KDEN Live, which uses the Melt framework as a backend.

In Section~\ref{section:comparisonMelt}, the results of video colour grading using the locally installed Melt framework and the implementation with JIT were compared. 
The results in this comparison display significant differences, even when no filter is applied. 
The results of the comparison with the original video frame and without a filter application reveal that the Accurate Player plays back the original video more accurately than the Melt framework.
These differences in the video representation without a filter application make it difficult to evaluate additional data regarding the application of filters using this method. The cause for the differences of the video frames by the Melt framework remains unclear.
One possible explanation for these differences in the extracted frames could be rooted in the system environment potentially influencing the evaluation of the comparison on different platforms. The differences could be caused by software dependencies influencing the Melt framework or its video player.



Additionally, the comparison with KDEN Live did not show the same results as the filter application in the Accurate Player either. The results of the visual comparisons, functionalities and colour subtractions lead to the assumption that the KDEN Live effect \textit{RGB adjustment} is not using the Melt filter \texttt{avfilter.colorbalance} to adjust the RGB values but a different filter.
The comparison revealed the different options and functionalities that arise with the use of different Melt filters and techniques, but those techniques make KDEN Live unsuitable for further comparisons with the implementation of this thesis project.


Those significant differences in the compared frames show the complexity of the topic of colour adjustment and representation. 
For this thesis project, all tests were run on the same device to ensure consistency. 
Ensuring a consistent output across various platforms and devices adds even more complexity to the task of colour representation and video streaming.
%
Additionally it was revealed, that there are many different ways to achieve similar results or functionalities in the topic of video colour grading. Many parameters and aspects have to be taken into consideration. This confirms the necessity for user-sided colour adjustment for a more customizable video streaming experience and satisfying results. 
	
	
	
\end{document}