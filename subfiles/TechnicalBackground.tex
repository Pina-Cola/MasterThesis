\documentclass[../MasterThesis.tex]{subfiles}
\graphicspath{ {./assets/images/} }


%----------------------------------------------------------------------------
%----------------------------------------------------------------------------

\begin{document}
	
	

%---------------------------------------------------------------------------
\newpage
%---------------------------------------------------------------------------

\section{Technical Background} \label{subsection:technicalbackground}

%-------------------------------------------------------------------------------------------------------
\subsubsection*{Video Transcoding} 

Transcoding is the conversion of one digital data format into another.~\cite{transcoding}







%-------------------------------------------------------------------------------------------------------
\subsubsection*{Video Compression with h.264} 
% Explain WebRTC

\begin{CountingDefinition}[h.264]{def:h264}
	
	h.264 refers to a video compression standard. In the context of real-time communication, h.264 is one of the video codecs that can be used to compress and decompress video streams.
	
\end{CountingDefinition}







%-------------------------------------------------------------------------------------------------------
\subsubsection*{WebRTC} 
% Explain WebRTC

\begin{CountingDefinition}[WebRTC]{def:WebRTC}
	
	Web Real-Time Communication (WebRTC), is a free, open-source project that provides web browsers and mobile applications with real-time communication. It enables direct communication between browsers or applications, allowing for peer-to-peer communication without the need for intermediary servers in certain scenarios.
	
\end{CountingDefinition}

WebRTC supports various video codecs, and h.264 is popular due to its efficiency in compressing video data while maintaining good quality. It is widely used for video conferencing, streaming, and other real-time communication applications. The video streams exchanged between the browser and the backend are encoded and decoded using h.264.






%-------------------------------------------------------------------------------------------------------
\subsubsection*{Audio Video Interleave} 

Audio Video Interleave (AVI) is a file format that can contain audio and video information to allow synchronized playback of audio and video components. 
It is a container format, which means that it can contain multiple streams of audio and video data, along with other multimedia data such as subtitles.~\cite{avi}












%-------------------------------------------------------------------------------------------------------
\subsubsection*{Named Pipe} 

A named pipe is a type of interprocess communication (IPC) mechanism. It provides a way for processes to pass data to each other and it is bidirectional. 
Named pipes follow a First-In-First-Out (FIFO) structure: The first data written into the pipe is the first data to be read. This maintains a sequential order for data transmission.
A traditional pipe is \textit{unnamed} and terminates when its process terminates. A named pipe can last beyond the life of the process, as long as the system is running.~\cite{namedpipe}









%-------------------------------------------------------------------------------------------------------
\subsubsection*{REST} 


Representational State Transfer (REST) is an architectural style for designing networked applications. A REST Application Programming Interface (API) exposes a set of endpoints (URLs) that allow communication between different software systems over the internet and it uses the hypertext transfer protocol (HTTP) methods.~\cite{IEEE_Rest, webservice, Nodejs_Rest}











%-------------------------------------------------------------------------------------------------------
\subsubsection*{JIT} 
%
\begin{CountingDefinition}[JIT]{def:JIT}
	
	JIT (Just-In-Time), in the context of video files for streaming, refers to a dynamic software solution employed for real-time video transformation on a server. It is used for on-the-fly conversion of video files, optimizing them for streaming. The process includes the transformation of a video file, potentially one with non-web-friendly formats, into a more suitable format for the playback in web browsers.
	
\end{CountingDefinition}






	
	
	
	
\end{document}